\section{概要}
西谷研究室での数値計算を用いた研究で多く使われるのが,MapleとRubyである.Rubyではできない計算をMapleにさせていたが,別々のソフトウェアを使うよりもRubyのみで完結させるためにインターフェースライブラリを開発することにした.今回の研究では,Mapleのコマンドライン実行される計算エンジン部に着目し,そこに働きかけて操作した.整数論関係から7つ,行列から6つの関数を選抜し実装した上で,関数に応じたwrapperを作り正しい出力型を取れるようにした上で,動的メソッドの実装後と実装前でどちらのプログラムがどう良いか比較した.そして,正しく計算できているか,またはRubyのみでの計算よりも数式処理能力が優れているかを検証し,成功した.今後は今回実装しなかった関数や,Mapleの特性である綺麗なグラフを出力できるような関数を実装できれば良いと思う.

