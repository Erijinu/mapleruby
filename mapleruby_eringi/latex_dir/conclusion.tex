\section{おわりに}
\subsection{今後の課題}
今回は限られた関数のみを選抜して実装したが,他にもたくさんの数学関数がMapleには用意されている.RSA暗号のプログラムをmaplerubyで実装した際に直接Maplerubyクラスに送ることで対応した等式の解を出力するevalのようなよく使われるであろう関数へ対応させる事や,累乗や四則混合の簡単な計算は現状直接Maplerubyクラスに送るような形をとっているため,そちらについてもうまく対応させたい.他にも桁数が大きな数値はそもそもRubyの変数が扱いきれない場合も考えられるので,そこにうまく対処できるような関数が欲しい.

また,Mapleが綺麗にグラフを描画できる数式処理ソフトウェアであることを利用して,maplerubyもグラフ描画に対応させる.MapleはCUI版でのグラフがかなり見にくく,二次元ならまだしも三次元になると何が何だか分からないグラフになるため,画像としてのグラフを出力できるような関数を実装する.

