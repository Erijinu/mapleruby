\section{おわりに}
\subsection{今後の課題}
\subsubsection{関数の充実化}
今回は限られた関数のみを選抜して実装したが,他にもたくさんの数学関数がMapleには用意されている.
RSA暗号のプログラムをmaplerubyで実装した際に直接Maplerubyクラスに送ることで対応した等式の解を出力するevalや累乗などの
頻繁に使われるであろう関数についてできるだけ対応させる.
他にも桁数が大きな数値はそもそもRubyの変数が扱いきれない場合も考えられるので,そこにうまく対処できるような関数が欲しい.

\subsubsection{グラフの描画}
Mapleが綺麗にグラフを描画できる数式処理ソフトウェアであることを利用して,maplerubyもグラフ描画に対応させる.
MapleはCUI版でのグラフがかなり見にくく,二次元ならまだしも三次元になると何が何だか分からないグラフになるため,
画像としてのグラフを出力できるような関数を実装する.

