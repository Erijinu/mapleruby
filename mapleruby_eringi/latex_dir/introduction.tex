\section{第一章 序論}
\subsection{Mapleとは}
Mapleは,1980年にカナダ・ウォータールー大学で生まれた数式処理技術をコアテクノロジーとして持つ
科学・技術・工学・数学(STEM : Science, Technology, Engineering and Mathematics)に関する
統合的計算環境である[Maple].
特徴として,たくさんの数学関数が用意されていること,
大きな桁数の計算が可能であること,グラフの描画が簡単であり,かつ3次元のグラフの描画にも
対応していることなどが挙げられる.
数式を入力するだけで簡単に解を得ることができることから,多くの場で用いられている.

\subsection{Rubyとは}
Rubyは,まつもとゆきひろ氏によって開発されたオブジェクト指向スクリプト言語である.
他にもテキスト処理に適した正規表現や高階関数,ガベージ・コレクションなどの特徴を持っている.

\subsubsection{RubyとPython}
Pythonは,Rubyと同じオブジェクト指向のスプリクト言語である.この2つは
度々比較され,どちらが優れているのかを議論されてきた.2つのスプリクト言語の
決定的な違いは何を得意としているかである.RubyはWeb分野を得意とするのに対し,Pythonは
数値計算やビッグデータを得意としている.逆にRubyは数値計算には弱く,PythonはWeb分野には弱い.
もちろんこの議論に答えはなく,本来は自分が何を
目的としたプログラムを作るのかで使い分けるのが理想だろう.しかし,Rubyを使い慣れている
人が数値計算をするためだけにPythonを勉強し直すよりも,Ruby上で数値計算ができる方が良いのは
明らかである.

\subsection{開発の背景}
先述の通り,Rubyは数値計算関連の環境整備が遅れており,Ruby上で高等な関数,
例えば,大きな素数を求めたり,最小公倍数を求めるなど
の処理を行うのが難しい.
一方で,Ruby以外の数式処理ソフトウェアなどを立ち上げて,
別々に作業したり,慣れない別の言語を勉強し直したりするよりもRubyのみでプログラミングする方が,
開発速度の格段の向上が期待できる.
そこで本研究では,数式処理ソフトウェアの1つであるMapleをRuby上で呼び出し,
Mapleに計算をさせて,その結果をRubyが取得するインターフェースライブラリの開発を目的とする.

