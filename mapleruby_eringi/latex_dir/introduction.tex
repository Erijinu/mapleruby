\section{序論}
Rubyは,まつもとゆきひろ氏によって開発されたオブジェクト指向スクリプト言語である.他にもテキスト処理に適した正規表現や高階関数,ガベージ・コレクションなどの特徴を持っている.フリーソースソフトウェアであるため,誰でも自由に使用することが可能である.

一方,数値計算分野においてはPythonが多用される.PythonはRubyと同じオブジェクト指向のスプリクト言語である.この2つは度々比較され,どちらが優れているのかを議論されてきた.2つのスプリクト言語の決定的な違いは何を得意としているかである.RubyはWeb分野を得意とするのに対し,Pythonは数値計算やビッグデータを得意としている.逆にRubyは数値計算には弱く,PythonはWeb分野には弱い.もちろんこの議論に答えはなく,本来は自分が何を目的としたプログラムを作るのかで使い分けるのが理想だろう.しかし,Rubyを使い慣れている人が数値計算をするためだけにPythonを勉強し直すよりも,Ruby上で数値計算ができる方が良いのは明らかである.

西谷研究室では数値計算を用いた研究を行っている.その研究で度々使われるのが数式処理ソフトウェアの1つであるMapleである.Mapleは,1980年にカナダ・ウォータールー大学で生まれた数式処理技術をコアテクノロジーとして持つ科学・技術・工学・数学(STEM : Science, Technology, Engineering and Mathematics)に関する統合的計算環境である[1].特徴として,たくさんの数学関数が用意されていること,大きな桁数の計算が可能であること,グラフの描画が簡単であり,かつ3次元のグラフの描画にも対応していることなどが挙げられる.数式を入力するだけで簡単に解を得ることができることから,多くの場で用いられている.

一方でソフト開発にはRubyを用いている.Rubyは数値計算関連の環境設備が遅れているため,Rubyのみで高等な関数,例えば,大きな素数を生成したり,最小公倍数を求めるなどの処理を行うのが難しい.また,扱える数値の桁数が計算内容によっては足りないということも考えられる.一方で,Ruby以外の数式処理ソフトウェアなどを立ち上げて,別々に作業したり,慣れない別の言語を勉強し直したりするよりもRubyのみでプログラミングする方が,開発速度の格段の向上が期待できる.そこで本研究では,MapleをRuby上で呼び出し,Mapleに高等な関数や桁数の大きな数値を用いた計算をさせて,その結果をRubyが取得するインターフェースライブラリの開発を目的とする.

