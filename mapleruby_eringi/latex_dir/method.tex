\section{第二章 手法}
\subsection{Mapleとの通信手法}
Mapleは一般的に,グラフや数式の綺麗な出力や,数式の入力を
初心者が直感的におこなえるようにJavaで作られたGUIを使って実行する.
それとは別にcommand lineで実行される計算エンジン部が用意されている.
そこで,開発するRubyライブラリでは,このエンジンに直接働きかけて操作する.
Rubyで外部コマンドを実行するgem libraryのsystemuを使って,
出力を得るようにしている.Ruby codeで要求コードを受け取った場合,
そのコードをtmp.mwに書き込む.それをMapleで実行し,結果をテキストファイルで
受けとることで出力を得る.

\subsection{Maple関数の類型化}
今回,数多く存在するMapleの数学関数の中から整数論と行列に関するものを選抜し実装した.
図1 実装した整数論に関する関数の役割と入出力
図2 実装した行列に関する関数の役割と入出力

\subsection{出力の切り替え}
Mapleから受け取ったままの出力は,値の前にスペースがたくさん入っていることや,
出力が String 型であることから,その数値を使って計算をするようにプログラミングしていた場合に
支障をきたす.このため,関数ごとに正しい型で出力できるようにwrapperを作る.
例えば,int 型で出力が欲しいものはexecをexec\_iから呼び出すことで対応する.
このようにbooleanやfloatといった出力型に応じて,exec\_b,exec\_fのように関数を増やしていく.

\subsubsection{出力の切り替えの例 - 整数論の場合}
まず,nextprimeを例にとると
\begin{quote}\begin{verbatim}
require "mapleruby/version"
require 'systemu'
require 'yaml'

class RMaple
	#整数論
	def nextprime(a)
		a = a.to_i
		p Mapleruby.new("nextprime(#{a})").exec_i
	end
end
class Mapleruby
	DATA_FILE=File.join(ENV['HOME'],'.mapleruby_rc')
	# Your code goes here...
	def initialize(maple_code)
		@maple_code = maple_code
		@src = get_env
		@maple_path=@src[:MAPLE_PATH]
	end
	def exec_i
		result = exec
		return result.to_i
	end
		def exec
		 code0=<<EOS
interface(quiet=true);
writeto("./result.txt");
#{@maple_code};
writeto(terminal);
interface(quiet=false);
EOS
		 File.write('tmp.mw',code0)
		 command="#{@maple_path} tmp.mw"
		 status,stdout,stderr=systemu command
		 status,stdout,stderr=systemu 'cat result.txt'
		# result=stdout
	  # print(result)
		 return stdout
	end
	def get_env
		begin
			file = File.open(DATA_FILE,'r')
		rescue => evar
			p evar
			print "no resource file for mapleruby.\n"
		end
		@src = YAML.load(file.read)
		file.close
		p @src
	end

end
\end{verbatim}\end{quote}
このようにexec\_iは,execを通った値をto\_iし,int型にしてから返すようになっている.
もし使われた関数がisprimeだった場合はexex\_bから返されるときに,boolean型のtrueとfalseが返されるようになっている.

\subsubsection{出力の切り替えの例 - 行列}
